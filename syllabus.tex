% Packages and document settings
\documentclass[11pt]{article} % Document type
\usepackage[a4paper,margin=1in]{geometry} % Page size, margins
\usepackage{fancyhdr} % Customize headers
\usepackage{enumitem} % Customize lists
\usepackage{booktabs} % Customize tables
\usepackage{hyperref} % Hyperlink support
\usepackage{url}      % URL formatting
\usepackage{xcolor}   % Color URLs
\hypersetup{
    colorlinks=true,
    linkcolor=blue,      
    urlcolor=blue,
    citecolor=blue,
}

% Load configuration values, such as course title, instructor name, etc..
% Configuration settings (Different for every class)
% -- Reusable constants for a specific course and section
% -- Examples are given for each setting, and are nonexhaustive
% -- Add more or remove some if appropriate for your own course

% Course Information
\newcommand{\coursenumber}{CSCI 26000}      % e.g. CSCI 49900
\newcommand{\coursesection}{02}             % Section number
\newcommand{\semester}{Summer 2024}         % Semester Year
\newcommand{\coursetitle}{Computer Architecture 2}     % Course Title
\newcommand{\modality}{In Person}  % Modality of the course, e.g., In-Person, Hybrid, Online Synchronous
\newcommand{\room}{North Bldg C107}      % Class location, e.g. HN C107 or Hunter North C107
\newcommand{\prerequisites}{CSCI 13500 and (CSCI 16000 or CSCI 24500) and MATH 15000 all with a grade of D or better} % Course prerequisites

% Instructor information
\newcommand{\instructor}{Charles Richards}   % Full name
\newcommand{\email}{cericha@hunter.cuny.edu}
\newcommand{\officehours}{By Appointment} % Office hours
\newcommand{\coursetime}{Mo We 5:45PM-8:53PM}    % Course meeting times, e.g. Mondays and Tuesdays from 5:00 PM - 8:00 PM

% Other information
\newcommand{\absences}{3} % Number of allowed absences without penalty



\setlength{\headheight}{14pt} % Header setting

% Set title, author, and date information
\title{\coursenumber\\\coursetitle}
\author{}
\date{\semester}

% Start syllabus Content
\begin{document}

% Header information
\pagestyle{fancy} 
\fancyhead{}
\lhead{\coursenumber} % Left header
\rhead{\semester} % Right header

% Display title
\maketitle

% Syllabus sections
\thispagestyle{fancy}

\section*{Instructor and Course Details}
% Example course detail section.
% Customize for your course - For example, comment out room if modality is
% Online-Synchronous
\begin{itemize}[label={},leftmargin=0pt]
    \item Course Time: \coursetime
    \item Modality: \modality % In-Person, Hybrid, Online, etc..
    % \item Room: \room % Comment out if course is online
    \item Instructor: \instructor
    \item Email: \email
    % \item Office Location: \officelocation
    % \item Office Phone: \officephone
    \item Office Hours: \officehours
\end{itemize}
\section*{Course Description}
% Example course description
This course is a chance for Computer Science majors to test their mettle on major projects. Working in small groups, each group will pitch their project, then design and implement it, using at least two platforms and at least three programming languages.
\section*{Prerequisites}
\prerequisites
% \section*{Textbook}
% % Examples given below for three textbooks for a course
\begin{itemize}
    \item Computer Organization and Design MIPS Edition 6th edition. David A. Patterson; John L. Hennessy. ISBN 9780128201091 - Make sure to get the MIPS version.
\end{itemize}
 % Specify if any or all textbooks are optionl
\section*{Grading}
% Example grading policy
Grades will be based on several aspects.
The project, worth 35\% of your grade, will involve pitching an idea, designing and presenting your design, and then implementing and programming your project. Each group will receive a project grade, and each student receives a weighted individual project grade.
You will work with a team of 2 to 4 and no more than 5 other students.
Periodic presentations and written reports, reporting on peers and progress, will account for 60\% of your grade.

The Participation, Effort and Contributions grade will be determined in part by my evaluation of your contributions, your class participation and in part by your team members' evaluation of your contributions.
Your final project grade is determined by the following calcuation: $(\mathcal{A}/5.0)*(\mathcal{P})$, where $\mathcal{A}$ is your Participation, Effort and Contributions grade, and $\mathcal{P}$ is your group project grade.

\textbf{You must receive a grade of (55/100) on your individual project grade to pass this course.} See below the complete breakdown:

\par\vspace{1cm}
% Grading Breakdown
\begin{tabular}{ @{}ll@{} }
    \toprule
        Area & Percentage \\
    \midrule
        Project: & 35\% \\
        Major Presentations (Expect 2): & 20\% \\
        Weekly Presentations: & 20\% \\
        Participation, Effort and Contributions: & 5\% \\
        Research and Technology Reports: & 10\% \\
        Periodic Peer Reports: & 10\% \\
        Total: & 100\% \\
    \bottomrule
\end{tabular}
\section*{Goals}
% https://www.hunter.cuny.edu/academicassessment/HowTo/AssessMyCourse/IdentifyCLOs

Students will complete a project that demonstrates their understanding of key topics covered in the CS minor, their ability to learn independently, and their skill in applying their knowledge to create concrete results. There are two other related goals for this course: The first is to assess the students’ mastery of the department’s stated learning goals. The second is to assess the department’s effectiveness in addressing these goals. For a complete list of these goals, see \href{http://www.hunter.cuny.edu/csci/for-students/learning-goals-for-hunter-college-students}{http://www.hunter.cuny.edu/csci/for-students/learning-goals-for-hunter-college-students}.



\section*{Attendance}
% Example grading policy
Attendance will be taken regularly and is expected each and every class meeting. Students are allowed up to \absences{} absences. Attendance is especially important for weekly progress reports and the presentations of the project.

\section*{Academic Integrity}
Hunter College regards acts of academic dishonesty (e.g., plagiarism, cheating on examinations, obtaining unfair advantage, and falsification of records and official documents) as serious offenses against the values of intellectual honesty. The College is committed to enforcing the CUNY Policy on Academic Integrity and will pursue cases of academic dishonesty according to the Hunter College Academic Integrity Procedures.
\section*{ADA Statement}
In compliance with the ADA and with Section 504 of the Rehabilitation Act, Hunter College is committed to ensuring educational access and accommodations for all its registered students. Hunter College’s students with disabilities and medical conditions are encouraged to register with the Office of AccessABILITY for assistance and accommodation. For information and appointment contact the Office of AccessABILITY located in Room E1214 or call (212) 772-4857 /or VRS (646) 755-3129.
\section*{Hunter College Policy on Sexual Misconduct}
In compliance with the CUNY Policy on Sexual Misconduct, Hunter College reaffirms the prohibition of any sexual misconduct, which includes sexual violence, sexual harassment, and gender-based harassment retaliation against students, employees, or visitors, as well as certain intimate relationships. Students who have experienced any form of sexual violence on or off campus (including CUNY-sponsored trips and events) are entitled to the rights outlined in the Bill of Rights for Hunter College.
\begin{enumerate}[label=\alph*.]
    \item Sexual Violence: Students are strongly encouraged to immediately report the incident by calling 911, contacting NYPD Special Victims Division Hotline (646- 610-7272) or their local police precinct, or contacting the College’s Public Safety Office (212-772-4444).
    \item All Other Forms of Sexual Misconduct: Students are also encouraged to contact the College’s Title IX Campus Coordinator, Dean John Rose (jtrose@hunter.cuny.edu or 212-650-3262) or Colleen Barry (colleen.barry@hunter.cuny.edu or 212-772-4534) and seek complimentary services through the Counseling and Wellness Services Office, Hunter East 1123.
\end{enumerate}
CUNY Policy on Sexual Misconduct Link: \url{http://www.cuny.edu/about/administration/offices/la/Policy-on-Sexual-Misconduct12-1-14-with-links.pdf}


\section*{Anti-Bullying and Cyberbullying Policy}
Bullying, cyberbullying, online hate, intimidation, threats, harassment, and pressure to share schoolwork are all forms of violence. CUNY holds a zero tolerance stance towards all such acts. The University is committed to prevention of any form of bullying, will respond promptly to threats and/or acts, and will protect victims of bullying from retaliation.  As a criminal matter, the New York Attorney General defines cyberbullying as the use of email, websites, instant messaging, chat rooms, text messaging and digital cameras to antagonize and intimidate others. Disrupting a teleconferencing platform (such as Zoom/Skype/Blackboard Collaborate Ultra) is a federal crime.

\section*{Course Schedule}
% Example course schedule
\renewcommand{\arraystretch}{1.5}
\begin{tabular}{ | l | l |  }
    \hline
    Week 1: & January 31, 2024 - Introductions, Syllabus and Course Overview \\
    \hline
    Week 2: & etc.... \\
    \hline
    Week 3: & etc... \\
    \hline
    Week 4: & etc... \\
    \hline
    Week x: & ... \\
    \hline
    Final & May 22, 2024 - Final Presentations - Peer Report Due \\
    \hline
\end{tabular}
\\ \\
\textit{Note: Schedule is tentative and subject to change}

    
\section*{Syllabus Change Policy}
% Example syllabus change policy
Except for changes that substantially affect implementation of the evaluation (grading) statement, this syllabus is a guide for the course and is subject to change with advance notice. Students will find out about changes to the syllabus via class attendance or by electronic means.


\end{document}